\documentclass[titlepage]{article}
\title{Final Assignment:\\Integration of Tools and Practices}
\author{Mohammad Yavari}
\begin{document}
\maketitle
\tableofcontents
\pagebreak
\section{Git and GitHub:}
\subsection{Repository Initialization and Commits}
on GitHub, New Repository, Give it a Name, Clone it into Your PC (e.g. using the https link and git clone command), Create a main.tex file and .github/workflows/main.yml (further explained).
\subsection{ GitHub Actions for LaTeX Compilation}
first I did'nt know about the tags so I tried to do it on every push and remove the tag condition but when I discovered the releases that are created automatically and went through some difficulties and problems
I/ learned tags and I'm happy about it!\newline
you simply need to use the command git tag after the commit you want to be complied and then push using git push origin tagname\newline
(I did'nt changed the main.yml file) 
\section{Exploration Tasks}
\subsection{Vim Advanced Features}
\begin{enumerate}
\item Multiple Cursors with Visual Mode:\\\\

Vim allows you to use visual mode to create multiple cursors and edit multiple occurrences simultaneously.\\\\

Use Ctrl+v to enter visual block mode.\\
Move to select the desired columns or lines.\\
Press I to insert text simultaneously on all selected lines.\\
Press Esc to apply changes.
\item Folding:\\\\

Vim supports code folding, allowing you to collapse and expand sections of your code. This is beneficial for navigating large files or focusing on specific parts of the code.\\\\

zf followed by a motion: Create a fold.\\
zo and zc: Open and close folds, respectively.\\
:set foldmethod=indent: Automatically fold based on the indentation.\\
\item Marks and Jumps:\\\\

Marks allow you to bookmark a location in a file. You can set a mark with m{letter}, and then jump to that mark with `{letter}.\\\\

ma: Set mark a at the current cursor position.\\
`a: Jump to the position of mark a.\\
Jumps can be used to quickly navigate between different parts of your file. Vim also has a "jump list" (:jumps) that keeps track of recent cursor movements.\\
\end{enumerate}

\end{document}